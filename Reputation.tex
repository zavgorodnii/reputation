\documentclass[11pt]{article}
\usepackage[margin=1in]{geometry}          
\usepackage{graphicx}
\usepackage{amsthm, amsmath, amssymb}
\usepackage{setspace}\onehalfspacing
\usepackage[loose,nice]{units}
 
\title{SONM User Reputation Model}
\author{Andrei Zavgorodnii, Anastasia Ashaeva}
\date{Jan, 2017}
 
\begin{document}

\maketitle
\tableofcontents
 
\section{Overview}

This document describes the reputation system that is used to characterise agents in the SONM network. Motivation: lost profit for sellers, lost time for buyers, etc.

\section{Basic notions}

The SONM ecosystem consists of \textit{buyers} and \textit{sellers}. Buyers rent computational resources from sellers to run arbitrary \textit{tasks}; a deal is made for a specific resource configuration and a specific period of time (i.e., not per task).

When looking for a seller, buyer searches the \textit{marketplace}. Deals made via marketplace are called \textit{public} deals\footnote{See \ref{privateDeals} for \textit{private} deals}. When a deal is made, a certain amount of funds is reserved on both buyer's and seller's accounts. For buyer, it's the \textit{full cost} of the deal; for seller, it's a fraction of the full cost (either default or negotiated), hereinafter \textit{the deposit}.

Both buyers and sellers have \textit{reputation} that is based on their activity in the SONM network, i.e., on the outcomes of public deals they had. Any public deal can have three possible outcomes:

\begin{itemize}
\item \textbf{Mutual satisfaction.} Buyer is satisfied with the service provided by seller. Buyer's and seller's reputation increases in proportion to deal cost.
\item \textbf{Settled dispute.} Buyer is \textit{not} satisfied with the service provided by seller, seller admits its fault. Buyer pays seller in proportion to the time it held seller's resources, and seller returns buyer the deposit.
\item \textbf{Claim.} Buyer is \textit{not} satisfied with the service provided by seller, seller refuses to admit its fault. Both buyer and seller keep their money, but lose their ratings in proportion to deal's cost.
\end{itemize}

\subsection{Private deals} \label{privateDeals}

\end{document}